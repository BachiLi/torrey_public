\documentclass{article}
\usepackage{hyperref}
\usepackage{listings}
\usepackage{color}
\usepackage{xcolor}
\usepackage{geometry}
\usepackage{graphicx}
\usepackage{amsmath}
\usepackage{caption}
\usepackage{subcaption}
\geometry{margin=1in}
\pdfminorversion=6

\newcommand\TODO[1]{\textcolor{red}{TODO: #1}}

\newcommand\header[2]{
    \begin{center}
        {\large
        UCSD CSE 168 Assignment #1: \\
        \vspace{0.3cm}
        \Large
        #2}
    \end{center}
}

\definecolor{dkgreen}{rgb}{0,0.6,0}
\definecolor{gray}{rgb}{0.5,0.5,0.5}
\definecolor{mauve}{rgb}{0.58,0,0.82}
\lstset{frame=tb,
        aboveskip=3mm,
        belowskip=3mm,
        showstringspaces=false,
        columns=flexible,
        basicstyle={\small\ttfamily},
        numbers=none,
        numberstyle=\tiny\color{gray},
        keywordstyle=\color{blue},
        commentstyle=\color{dkgreen},
        stringstyle=\color{mauve},
        breaklines=true,
        breakatwhitespace=true,
        tabsize=2
}

\hypersetup{colorlinks=true}

\usepackage{xcolor}

\begin{document}

\header{3}{Textures, shading normals, area lights, and BRDFs}

In the previous homework, we mostly focused on increasing the \emph{geometric complexity} of our renderer and scenes. In this homework, we're going to implement a bunch of small features for our renderer to increase the \emph{material} and \emph{lighting complexity} of our scenes. We will still stay in the \href{https://raytracing.github.io/books/RayTracingTheNextWeek.html}{Ray Tracing - The Next Week (RTNW)} book for this homework.

\section{Textures}
So far, we assume constant color per object. This is kind of boring and real world objects are {\color{red}c}{\color{orange}o}{\color{green}l}{\color{blue}o}{\color{cyan}r}f{\color{magenta}u}{\color{violet}l}. The way computer graphics deal with colors on objects is through \href{https://en.wikipedia.org/wiki/Texture_mapping}{textures} (invented by Edwin Catmull in the 70s!): basically we wrap an image around a surface and define color on that image. There are two common ways to specify a texture: you can specify it \emph{procedurally} through a program, or you can use a raster image to represent the texture. We will implement the raster image texture, and the procedural texture will be bonus.

Go read \href{https://raytracing.github.io/books/RayTracingTheNextWeek.html#solidtextures}{Chapter 4} and \href{https://raytracing.github.io/books/RayTracingTheNextWeek.html#imagetexturemapping}{Chapter 6} of RTNW.

For the UV coordinates, we will use the same as the RTNW book for spheres. For triangles, there are two possiblities. First, the triangle mesh can come with its own UV coordinates per triangle vertex. You can access it through \lstinline{ParsedTriangleMesh::uvs} -- if the mesh contains UV coordinates, \lstinline{uvs} would be an array with the same size as \lstinline{positions}. To get the UV coordinates at each point, we interpolate from the three UV values from the vertices using the barycentric coordinates $(s, t)$: $u = (1 - s - t)u_0 + s * u_1 + t * u_2$.
Second, if the triangle mesh does not come with UV coordinates (\lstinline{uvs.size() == 0}), we will just use the barycentric coordinates $(s, t)$ as our UV.

For loading an image, you can use the \lstinline{imread3} function defined in \lstinline{image.h}.

As usual, go to the function \lstinline{hw_3_1} in \lstinline{hw3.cpp}. Extend your previous renderer in Homework 2.5 to render images with textures. Replace the constant colors in our shading with the color provided by the textures.

%\bibliographystyle{plain}
%\bibliography{refs}

\end{document}
